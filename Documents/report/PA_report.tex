\documentclass[a4paper,11pt]{report}

% General tools
\usepackage{etoolbox}

\usepackage[utf8]{inputenc}

% Fonts
\usepackage[T1]{fontenc}

% LaTeX modern fonts
\usepackage{lmodern}

% Sans serif
%\usepackage{tgheros}

% Serif
%\usepackage[bitstream-charter]{mathdesign}

% Monospace
%\usepackage{sourcecodepro}

% Language
\usepackage[english]{babel}
\usepackage{blindtext}

% Page style
\usepackage{fullpage} % page margins to 1.5cm
\usepackage{fancyhdr} % headers and footers

% Colors & graphics
\usepackage[table]{xcolor}    % colors
\usepackage[pdftex]{graphicx} % graphics importing

% Misc
\usepackage{titlesec} % section titles formatting
\usepackage{minted}   % code highlighting
\usepackage{lscape}   % landscape
\usepackage{tikz}     % charts in LaTeX
\usepackage{amsmath}  % better math
\usepackage{float}    % floats
\usepackage[small,justification=centering]{caption}
\usepackage{microtype} % typographic improvements

% Paragraphs
\usepackage{parskip}
\usepackage[defaultlines=3,all]{nowidow}

% Chapter titles
% Remove space before title
\titlespacing{\chapter}{0pt}{*-4}{*3}
% Remove "Chapter N" and use a sans-serif font
\titleformat{\chapter}[hang]{\bf\huge}{\thechapter}{1pc}{}
% Change chapter page style
\patchcmd{\chapter}{plain}{fancy}{}{}

% Tables
\usepackage{multirow}

% Cross-references
\usepackage{hyperref}


% Metadata
% --------
% ==============================
% Authors of the LaTeX template:
%   - Sylvain Julmy
%   - Marc Demierre
% ==============================

% Metadata for this report
% ------------------------
\newcommand{\School}{University of Applied Sciences Western Switzerland}
\newcommand{\Faculty}{MSE - Software Engineering}
\newcommand{\Place}{Lausanne}

% Course

\newcommand{\Title}{<Title>}

% Supervisors (professors)
\newcommand{\Supervisors}{Déruaz Vincent}

% Students
\newcommand{\Authors}{Déruaz Vincent}

%TOOLS
\newcommand{\todo}[1]{\textcolor{red}{TODO: #1}\PackageWarning{TODO:}{#1!}}

\newcommand{\Course}{Projet d'approfondissement}

% Header and footer
% -----------------
\pagestyle{fancy}
\lhead[]{\Course}
\chead[]{}
\rhead[]{\Place, \today}

\setlength{\headheight}{14pt}
\setlength{\headsep}{14pt}

\newcommand{\HRule}{\rule{\linewidth}{0.5mm}}

% Code styles for highlighting
% ----------------------------

% How to use (replace 'java' with language name):
% - code blocks:
%     \begin{javacode}
%     CODE
%     \end{javacode}
% - files:
%     full: \javafile{PATH}
%     extract: \javafile[startline=x, endline=y]{PATH}
% TODO: inline?

% Java
\newminted{java}{frame=single, framesep=6pt, breaklines=true, fontsize=\scriptsize}
\newmintedfile{java}{frame=single, framesep=6pt, breaklines=true, fontsize=\scriptsize}

% Scala
\newminted{scala}{frame=single, framesep=6pt, breaklines=true, fontsize=\scriptsize}
\newmintedfile{scala}{frame=single, framesep=6pt, breaklines=true, fontsize=\scriptsize}

% Python
\newminted{python}{frame=single, framesep=6pt, breaklines=true, fontsize=\scriptsize}
\newmintedfile{python}{frame=single, framesep=6pt, breaklines=true, fontsize=\scriptsize}

% Plain text
\newminted{text}{frame=single, framesep=6pt, breaklines=true, breakanywhere, fontsize=\scriptsize}
\newmintedfile{text}{frame=single, framesep=6pt, breaklines=true, breakanywhere, fontsize=\scriptsize}

% Document
% --------
\begin{document}

\begin{titlepage}
    \begin{center}

        % only works if a paragraph has started.
        \includegraphics[width=0.8\textwidth]{img/mse_logo}~\\[1.5cm]
        \textsc{\Large \School}\\[0.25cm]
        \textsc{\Large \Faculty}\\[1.5cm]

        % Title
        \HRule \\[0.4cm]
        { \huge \bfseries \Course \\[0.4cm] }
        \HRule \\[1.5cm]

        % Author and supervisor
        \begin{minipage}[t]{0.4\textwidth}
            \begin{flushleft} \Large
                \emph{Authors:}\\ \Authors
            \end{flushleft}
        \end{minipage}
        \begin{minipage}[t]{0.4\textwidth}
            \begin{flushright} \Large
                \emph{Supervisors:}\\\Supervisors
            \end{flushright}
        \end{minipage}~\\[1.5cm]

        \begin{center}
            \includegraphics[scale=0.7]{img/logo_hes-so}
        \end{center}

        \vfill

        % Bottom of the page
        {\large \Place, \today}

    \end{center}
\end{titlepage}

\tableofcontents


\chapter{Cahier des charges}
 

\chapter{Compte rendu}

\section{Semaine 24-01 fevrier}
\subsection{Réunion de prise de contact}
\todo PV --> recopier notes manuscrite ICI

\subsection{Lectures}
Lectures sur la biologie et début bio-informatique

\subsection{Machine Virtuel}
Apres quelques recherches sur les différent outils se trouvant dans les lectures, mise en place d'une machine virtuelle, Linux Ubuntu, avec l'installation des package necessaire.



Installation Anaconda

\begin{verbatim}
conda install biopython
conda install -c https://conda.anaconda.org/biocore scikit-bio
conda install basemap
conda install graphviz
conda install pyproj
conda install pillow

wget http://www.ormbunkar.se/aliview/downloads/linux/linux-version-1.17.1/aliview.install.run
sudo chmod +x aliview.install.run
sudo ./aliview.install.run 

wget http://www.clustal.org/download/current/clustalw-2.1-linux-x86_64-libcppstatic.tar.gz
tar -xzvf clustalw-2.1-linux-x86_64-libcppstatic.tar.gz

wget http://www.drive5.com/muscle/downloads3.8.31/muscle3.8.31_i86linux64.tar.gz
tar -xzvf muscle3.8.31_i86linux64.tar.gz

wget http://darwin.zoology.gla.ac.uk/~rpage/treeviewx/download/0.5/tv-0.5.1.tar.gz
tar -xzvf tv-0.5.1.tar.gz
cd tv-0.5.1

sudo apt-get install build-essential
sudo apt-get install libwxgtk2.8-dev

./configure

cd ..

http://weblogo.berkeley.edu/release/weblogo.2.8.2.tar.gz
tar -xzvf weblogo.2.8.2.tar.gz
\end{verbatim}

\subsection{Cahier des charges}

\section{Semaine 02-08 mars}
\subsection{Réunion}
\todo PV

\subsection{Lectures}

\subsection{Tutorial}

\end{document}

